%% ----------------------------------------------------------------
%% Thesis.tex -- MAIN FILE (the one that you compile with LaTeX)
%% ---------------------------------------------------------------- 

% Set up the document
\pdfminorversion=7
\documentclass[a4paper, 11pt, oneside]{Thesis}  % Use the "Thesis" style, based on the ECS Thesis style by Steve Gunn
\graphicspath{Figures/}  % Location of the graphics files (set up for graphics to be in PDF format)

% Include any extra LaTeX packages required
\usepackage[square, numbers, comma, sort&compress]{natbib}  % Use the "Natbib" style for the references in the Bibliography
\usepackage{verbatim}  % Needed for the "comment" environment to make LaTeX comments
\usepackage{vector}  % Allows "\bvec{}" and "\buvec{}" for "blackboard" style bold vectors in maths
\usepackage{acronym}
\usepackage[final]{pdfpages}
%\usepackage[hidelinks]{hyperref}
%\hypersetup{colorlinks=false}  % Colours hyperlinks in blue, but this can be distracting if there are many links.
\hypersetup{
    colorlinks,
    linkcolor={red!50!black},
    citecolor={blue!50!black},
    urlcolor={blue!80!black}
}
%% ----------------------------------------------------------------
\begin{document}
\frontmatter      % Begin Roman style (i, ii, iii, iv...) page numbering

% Set up the Title Page
\title{Machine Learning Based Restaurant Location Analysis in Hamburg}
\authors  {
			\texorpdfstring
            {\href{}{Hung Viet Hu}}
            {Hung Viet Hu}, 
			\texorpdfstring
            {\href{}{Sachin Kumar}}
            {Sachin Kumar}, 
			\texorpdfstring
            {\href{https://www.linkedin.com/in/vincent-meyer-zu-wickern-705b9610a/}{Vincent Meyer zu Wickern}}
            {Vincent Meyer zu Wickern}
            }
            
            
\addresses  {\groupname\\\deptname\\\univname}  % Do not change this here, instead these must be set in the "Thesis.cls" file, please look through it instead
\date       {\today}
\subject    {}
\keywords   {}

\maketitle
%% ----------------------------------------------------------------

\setstretch{1.5}  % It is better to have smaller font and larger line spacing than the other way round

% Define the page headers using the FancyHdr package and set up for one-sided printing
\fancyhead{}  % Clears all page headers and footers
\rhead{\thepage}  % Sets the right side header to show the page number
\lhead{}  % Clears the left side page header

\pagestyle{fancy}  % Finally, use the "fancy" page style to implement the FancyHdr headers

%% ----------------------------------------------------------------
% Declaration Page required for the Thesis, your institution may give you a different text to place here
%\Declaration{

\addtocontents{toc}{\vspace{1em}}  % Add a gap in the Contents, for aesthetics
I, Vincent Finn Alexander Meyer zu Wickern, declare that this thesis titled, 'Predictive Maintenance' and the work presented in it are my own. I confirm that:

\begin{itemize} 
\item[\tiny{$\blacksquare$}] This work was done wholly or mainly while in candidature for a research degree at this University.
 
\item[\tiny{$\blacksquare$}] Where any part of this paper has previously been submitted for a degree or any other qualification at this University or any other institution, this has been clearly stated.
 
\item[\tiny{$\blacksquare$}] Where I have consulted the published work of others, this is always clearly attributed.
 
\item[\tiny{$\blacksquare$}] Where I have quoted from the work of others, the source is always given. With the exception of such quotations, this paper is entirely my own work.
 
\item[\tiny{$\blacksquare$}] I have acknowledged all main sources of help.
 
\item[\tiny{$\blacksquare$}] Where the paper is based on work done by myself jointly with others, I have made clear exactly what was done by others and what I have contributed myself.
\\
\end{itemize}
 
 
Signed:\\
\rule[1em]{25em}{0.5pt}  % This prints a line for the signature
 
Date:\\
\rule[1em]{25em}{0.5pt}  % This prints a line to write the date

\clearpage  % Declaration ended, now start a new page

%% ----------------------------------------------------------------
% The "Funny Quote Page"
%\pagestyle{empty}  % No headers or footers for the following pages
%
%\null\vfill
%% Now comes the "Funny Quote", written in italics
%\textit{``Write a funny quote here.''}
%
%\begin{flushright}
%If the quote is taken from someone, their name goes here
%\end{flushright}
%
%\vfill\vfill\vfill\vfill\vfill\vfill\null
%\clearpage  % Funny Quote page ended, start a new page
%% ----------------------------------------------------------------

% The Abstract Page
\addtotoc{Abstract}  % Add the "Abstract" page entry to the Contents
\abstract{
\addtocontents{toc}{\vspace{1em}}  % Add a gap in the Contents, for aesthetics


}

\clearpage  % Abstract ended, start a new page
%% ----------------------------------------------------------------

\setstretch{1.5}  % Reset the line-spacing to 1.3 for body text (if it has changed)

% The Acknowledgements page, for thanking everyone
%\acknowledgements{
%\addtocontents{toc}{\vspace{1em}}  % Add a gap in the Contents, for aesthetics

%The acknowledgements and the people to thank go here, don't forget to include your project advisor\ldots

%}
%\clearpage  % End of the Acknowledgements
%% ----------------------------------------------------------------

\pagestyle{fancy}  %The page style headers have been "empty" all this time, now use the "fancy" headers as defined before to bring them back


%% ----------------------------------------------------------------
\lhead{\emph{Contents}}  % Set the left side page header to "Contents"
\tableofcontents  % Write out the Table of Contents

%% ----------------------------------------------------------------
%\lhead{\emph{List of Figures}}  % Set the left side page header to "List if Figures"
%\listoffigures  % Write out the List of Figures

%% ----------------------------------------------------------------
%\lhead{\emph{List of Tables}}  % Set the left side page header to "List of Tables"
%\listoftables  % Write out the List of Tables

%% ----------------------------------------------------------------
\setstretch{1.5}  % Set the line spacing to 1.5, this makes the following tables easier to read
\clearpage  % Start a new page
\lhead{\emph{Abbreviations}}  % Set the left side page header to "Abbreviations"
\btypeout{Abbreviations}
\chapter{Abbreviations}


%\listofsymbols{ll}  % Include a list of Abbreviations (a table of two columns)
%{
%% \textbf{Acronym} & \textbf{W}hat (it) \textbf{S}tands \textbf{F}or \\
%\textbf{LAH} & \textbf{L}ist \textbf{A}bbreviations \textbf{H}ere \\
%
%}
\begin{acronym}
\acro{alkis}[ALKIS\copyright]{Amtliches Ligenschaftskatasterinformationssystem (Authoritative Real Estate Cadastre Information System)}
\acro{metaver}[MetaVer\copyright]{MetadatenVerbund}
\end{acronym}

%% ----------------------------------------------------------------
%\clearpage  % Start a new page
%\lhead{\emph{Physical Constants}}  % Set the left side page header to "Physical Constants"
%\listofconstants{lrcl}  % Include a list of Physical Constants (a four column table)
{
% Constant Name & Symbol & = & Constant Value (with units) \\
%Speed of Light & $c$ & $=$ & $2.997\ 924\ 58\times10^{8}\ \mbox{ms}^{-\mbox{s}}$ (exact)\\

}

%% ----------------------------------------------------------------
%\clearpage  %Start a new page
%\lhead{\emph{Symbols}}  % Set the left side page header to "Symbols"
%\listofnomenclature{lll}  % Include a list of Symbols (a three column table)
{
% symbol & name & unit \\
%$a$ & distance & m \\
%$P$ & power & W (Js$^{-1}$) \\
%& & \\ % Gap to separate the Roman symbols from the Greek
%$\omega$ & angular frequency & rads$^{-1}$ \\
}
%% ----------------------------------------------------------------
% End of the pre-able, contents and lists of things
% Begin the Dedication page

\setstretch{1.3}  % Return the line spacing back to 1.3

%\pagestyle{empty}  % Page style needs to be empty for this pagef
%\dedicatory{For/Dedicated to/To my\ldots}

\addtocontents{toc}{\vspace{2em}}  % Add a gap in the Contents, for aesthetics


%% ----------------------------------------------------------------
\mainmatter	  % Begin normal, numeric (1,2,3...) page numbering
\pagestyle{fancy}  % Return the page headers back to the "fancy" style

% Include the chapters of the thesis, as separate files
% Just uncomment the lines as you write the chapters

\chapter{Introduction}
\lhead{\emph{Introduction}}

\chapter{Elements of a restaurant location analysis}
\lhead{\emph{Elements of a restaurant location analysis}}


\chapter{Data Sources}
\lhead{\emph{Data Sources}}


\section{Transparency Portal Hamburg}
As the first German federal state, Hamburg enacted a transparency law on October 6, 2012 \cite{Murjahn.2016}. Opposed to a right to request information, which all citizens had until this date, a new duty to inform the public was laid upon the state’s administration offices. All information that would fall under this law, now had to to be published in a freely available standard format on a centered storage of information. The single pieces of information, which would fall under the law, varied highly in precision and the comprehensive term of ``geodata'' was requested opposed to precise datasets of geodata. A legal interpretation was worked out for all requested points and a plan for the release of geodata was designed consisting of the basic data for measurement admistration and the technical geodata for special administration offices. The transparency law granted a period of two years for the technical implementation.

In October 2014, the ``Transparency Portal'' (http://transparenz.hamburg.de/) as the major component of the implementation of the transparency law was released \cite{Murjahn.2016}. With this portal, the Hamburg citizens have a multitude of data and documents available that was prior only available to Hamburg's administration. One important focus was the release of geodata that was even before the law in preparation for an ``Open GeoData'' model. In this ``Open GeoData'' model, geodata was split into two groups of data sets, one group extractable with little effort, but free to the public and expected with a high use, and another group with expected high demand and high revenue on the sale of this data. For this second group of datasets, more effort with new measurements had to be arranged. With the transparency law in place, all datasets were merged into the Transparency Portal and yielded a much higher download count than the count of dataset sales before the portal was active. The Transparency Portal uses a standardized meta data repository called the \ac{metaver} in collaboration with other German federal states.

\section{Yelp}
 
 \chapter{Analysis Methods}
\lhead{\emph{Analysis Methods}}

\section{Random Forest Regression}

\section{Performance Measures}

\chapter{Data Extraction}
\lhead{\emph{Data Extraction}}

\section{Yelp Restaurant Data Extraction}

\section{Hamburg District Map}

Since usually cities raise important figures in aggregation per admistrative area, in Hamburg being the single city districts, these administrative areas should be imported into QGIS to be able to link single restaurants to an administrative area and hence figures that could be important as dependent variables to predict restaurant success. The borders administrative areas are taken from a dataset  of the Transparency Portal called ``ALKIS Verwaltungsgrenzen Hamburg'' \cite{LandesbetriebGeoinformationundVermessung.28.02.2018}. This dataset is available in multiple dataformats, is reported to have a 0\% data deficit and a precision of 0.1 meters. It is part of the \ac{alkis}, a digital combination of the real estate book information and a real estate map \cite{ALKIS2019}.

\section{Social Values}

\section{Proximity to Water}

\chapter{Machine Learning}
\lhead{\emph{Machine Learning}}

\section{Exploratory Data Analysis}

\section{Data Preprocessing}

\subsection{Handling of Missing Values}

\subsection{Feature Subset Selection}

\subsection{Dimensionality Reduction}

\section{Data Analysis}


\chapter{Results and Discussion}
\lhead{\emph{Results and Discussion}}

\section{Results}

\section{Discussion}

\chapter{Conclusion}
\lhead{\emph{Conclusion}}  



%% ----------------------------------------------------------------
% Now begin the Appendices, including them as separate files

%\addtocontents{toc}{\vspace{2em}} % Add a gap in the Contents, for aesthetics

%\appendix % Cue to tell LaTeX that the following 'chapters' are Appendices

%\input{Appendices/AppendixA}	% Appendix Title

%\input{Appendices/AppendixB} % Appendix Title

%\input{Appendices/AppendixC} % Appendix Title

\addtocontents{toc}{\vspace{2em}}  % Add a gap in the Contents, for aesthetics
\backmatter

%% ----------------------------------------------------------------
\label{Bibliography}
\lhead{\emph{Bibliography}}  % Change the left side page header to "Bibliography"
\bibliographystyle{unsrtnat}  % Use the "unsrtnat" BibTeX style for formatting the Bibliography
\bibliography{Bibliography}  % The references (bibliography) information are stored in the file named "Bibliography.bib"

\end{document}  % The End
%% ----------------------------------------------------------------